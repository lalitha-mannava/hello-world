%---------change this every homework
\def\yourid{svm5pc}
\def\collabs{}
\def\sources{Cormen, et al, Introduction to Algorithms}
% -----------------------------------------------------
\def\duedate{Tuesday, September 3, 2019 at 11p}
\def\duelocation{via Collab}
\def\hnumber{0}
\def\course{{cs4102 - algorithms - fall 2019}}%------
%-------------------------------------
%-------------------------------------

\documentclass[10pt]{article}
\usepackage[colorlinks,urlcolor=blue]{hyperref}
\usepackage[osf]{mathpazo}
\usepackage{amsmath,amsfonts,graphicx}
\usepackage{latexsym}
\usepackage[top=1in,bottom=1.4in,left=1.25in,right=1.25in,centering,letterpaper]{geometry}
\usepackage{color}
\definecolor{mdb}{rgb}{0.1,0.4,0.52} 
\definecolor{cit}{rgb}{0.05,0.2,0.45} 
\pagestyle{myheadings}
\markboth{\yourid}{\yourid}
\usepackage{clrscode}

\newenvironment{proof}{\par\noindent{\it Proof.}\hspace*{1em}}{$\Box$\bigskip}
\newcommand{\handout}{
   \renewcommand{\thepage}{Homework \hnumber - page \arabic{page}}
   \noindent
   \begin{center}
      \vbox{
    \hbox to \columnwidth {\sc{\course} \hfill}
    \vspace{-2mm}
    \hbox to \columnwidth {\sc due \MakeLowercase{\duedate} \duelocation\hfill {\Huge\color{mdb}H\hnumber:\yourid}}
      }
   \end{center}
   \vspace*{1mm}
   \hrule
   \vspace*{1mm}
    {\footnotesize \textbf{Collaboration Policy:} You are encouraged to collaborate with up to 4 other students, but all work submitted must be your own independently written solution. List the computing ids of all of your collaborators in the \texttt{collabs} command at the top of the tex file. Do not seek published or online solutions for any assignments. If you use any published or online resources (which may not include solutions) when completing this assignment, be sure to cite them. Do not submit a solution that you are unable to explain orally to a member of the course staff.
   \vspace*{1mm}
    \hrule
    \vspace*{2mm}
    \noindent
    \textbf{Collaborators}: \collabs\\
    \textbf{Sources}: \sources}
    \vspace*{2mm}
    \hrule
    \vskip 2em
}
\newcommand{\solution}[1]{\medskip\noindent\textbf{Solution:}#1}
\newcommand{\bit}[1]{\{0,1\}^{ #1 }}
%\dontprintsemicolon
%\linesnumbered
\newtheorem{problem}{\sc\color{cit}problem}
\newtheorem{practice}{\sc\color{cit}practice}
\newtheorem{lemma}{Lemma}
\newtheorem{definition}{Definition}


\begin{document}
\thispagestyle{empty}
\handout

%----Begin your modifications here

\begin{problem}Reductio ad absurdum\end{problem}

\begin{quote}
\textit{Reductio ad absurdum}, which Euclid loved so much, is one of a mathematician's finest weapons. It is a far finer gambit than any chess play: a chess player may offer the sacrifice of a pawn or even a piece, but a mathematician offers the game. [Excerpt from {\em A Mathematician's Apology}, G.H. Hardy, 1940, p. 94]  
\end{quote}

\noindent Learn how to write math and construct proofs by reproducing the proof below. You will need to use the \verb|eqnarray| or \verb|align| environment, as well as the \verb|eqnarray*| or \verb|align*| environment.  Note the reference in red, which should refer correctly to the equation.

\begin{definition}
A rational number is a fraction $\frac{a}{b}$ where $a$ and $b$ are integers. 
\end{definition}

\noindent Show $\sqrt{2}$ is irrational.

\begin{proof}
\end{proof}

\begin{problem} It was the best of times, it was the worst of times...\end{problem}
    Include a passage from your favorite book, including a citation.


\begin{problem}Sketchings\end{problem}
Learn how to include drawings in your documents with the \verb|\includegraphics{file}| command by submitting a caricature of professor Hott or professor Wu.


\end{document}

